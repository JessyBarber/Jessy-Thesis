\begin{comment}
\section{Introduction}

\section{Bridge Health Monitoring}

\section{Wireless Sensor Networks and LoRa}
Wireless Sensor Networks (WSNs) are simple, low-cost networks that primarily consist of nodes and a base station \cite{WSN-WaterQual}. WSN nodes usually comprise of some sensing or measuring capability and relay this information via uplink to a base station for processing and then to a network server. Innovating many field of industry and research, these distributed networks of nodes have been valuable in many contexts. For example, the use of ZigBee communication technology for air pollution monitoring \cite{ZigBeeAirPolution} and the use of Bluetooth for communication between end-devices measuring temperature, luminance, carbon dioxide and humidity for energy-saving establishments \cite{BTenergySaving}. Although these WSNs have worked in the past, the future of this technology lies in developing systems that have high scalability and range, something that ZigBee and Bluetooth inherently lack. Cellular and satelite .... space study ... somewhere 

Long Range (LoRa) technology was introduced 

\section{The Internet of Things and LoRaWAN}
The configuration of WSNs have typically been a deployment of Wireless Personal Area Network (WPAN) or Low Power Wide Area Network (LPWAN) standards, where nodes are setup in a mesh layout using a short-range communication protocol such as ZigBee and Bluetooth \cite{WSN-WaterQual}. The main implication with these protocols is that the mesh implementation inherently bottlenecks scalability due to exponentially increasing network requirements and power consumption \cite{IOTandLORAWAN-SmartFarm}. Long Range Wide Area Network (LoRaWAN) is a solution to implementing an LPWAN system with minimal complexity and scalability whilst also being low power. LoRa by definition is a chirp spread spectrum (CSS) modulation technique developed by Cycleo offering a Medium Access Control (MAC) layer protocol and operates on the `licence-free region-dependent industrial, scientific, and medical (ISM) frequncy bands' \cite{IOTandLORAWAN-SmartFarm}. In Australia the operational ISM band for LoRa is between 915 and 928 MHz. LoRaWAN is the ideal technology for agritulcutral and regional purposes due to its long range, low power and long lifetime. 
\end{comment}