\section{Effectiveness of LoRa Node}
The beam laboratory and Griffith footbridge testing demonstrated the effectiveness of LoRa nodes in structural health monitoring as per the following points:
\subsection{FFT Accuracy and Device Calibration} 
The laboratory beam testing confirmed the effectiveness of the FFT computation within the software and the ability to accurately identify peak maximum frequencies. This accuracy was verified with comparison to the first mode fundamental frequency achieved in the Strand7 FE simulation as both achieved peak frequency of 2.4 Hz. Additionally, the device's functionality remained consistent when powered by a battery pack and operating under a lower reference voltage. This illustrates the device's adaptability to different power sources and confirms the successful calibration of the accelerometer which is essential for real-world application and deployment.
\subsection{Low Power}
A multimeter was used to measure the current draw of the device during the FFT computation and LoRa packet transmission when using the battery pack. With 3.0 V input voltage from the two triple A batteries, the current draw of the device was approximately 23.8 mA. The batteries were listed at 1250 mAh which means that prototype two was capable of supporting 52.5 hours of deployment with constant computation load. This is a long lifetime for a prototype, highlighting the low-power nature of LoRa nodes, which can be improved through various metrics which are discussed in the recommendations section. 
\subsection{Noise Thresholds and Response}
Adjustments to the noise thresholds in the acceleration processing greatly influenced the frequency response of the system. Tailoring these values allowed the system to align closely with the first mode fundamental frequency from the FE simulation in the prototype one laboratory beam test, and within 12.8\% of the first mode flexural frequency in the prototype two bridge test. Further trial and error testing can result in a much higher accuracy by further fine tuning the noise threshold parameters. 
\subsection{Impact of Pedestrian Load}
The sensitivity of prototype two to variations in pedestrian load indicate its capability to monitor and interpret external influences on the structure. This responsiveness can further the system's ability to analyse structural behaviour under varying conditions. The impact to the structural health of the bridge can be observed by logging the first mode flexural frequency over long periods of time in a future deployment. This can give indication as to how significant the impact of high pedestrian load or an accident has on the bridge's structural integrity. 

\section{LoRaWAN Signal Strength}
The main limitations of this project exist in the communication range and signal strength of the prototype two deployment. Some characteristics of the LoRaWAN signal strength do show promising prospects for future design revisions which are discussed in the following points:

\subsection{Communication Range}
Prototype one test three confirmed a successful communication range of 200m, showcasing the system's potential for large-scale structural monitoring. However, there were issues involved in the test setup for prototype two. Initially the base station was deployed in a nearby building but was completely closed off to the outside by thick windows. After three to four successful packets, the connection was dropped. This indicates that the prototype two deployment was highly susceptible to obstacles and environmental factors.\\\\
This susceptibility is a result of the antenna, gateway and the environment. The antenna is rated at a gain of -1 dBi in the frequency band of 824 MHz - 960 MHz which restricts the directionality and range of the signal. The gateway was designed for indoor use and so it is likely that the gateway antenna was not designed to penetrate many obstacles such as the thick wall and windows of the enclosed building of the initial base-station deployment. When conducting the range test for prototype one, the door of the laboratory was left open so the test was not isolated from the outside environment. A combination of these factors resulted in a poor communication range for the prototype two implementation, but the exact maximum range for the node to gateway communication in a non-isolated environment is yet to be determined. A antenna with higher gain for the node device and the use of an outdoor gateway would be imperative to achieving a much higher communication range that is more characteristic of LoRaWAN specifications. 
\subsection{Antenna Orientation and Placement}
The results from the prototype two testing indicate that the physical placement and orientation of the antenna on the node device influenced the signal strength and frequency response. Improved SNR and RSSI values were noted when the device was flat and the antenna was directed towards the base station. The enclosure was not specifically designed with this directionality in mind which is an aspect to consider for future revisions.
\subsection{Data Corruption and Signal Resilience}
Even in periods of reduced signal strength, the prototype system was capable of resuming normal functionality which highlights the robustness of LoRaWAN technology. Even with these periods of reduced signal strength, there was no packet loss due to the resilience of the protocol and its ability to transmit with RSSI values as low as -120 dBm \ref{LoRa-SNR-RSSI}. The presence of a strong peak in average max acceleration coinciding with a significant signal drop in test three suggests potential data corruption or noise interference which warrants further investigation. 

\section{Recommendations}
Moving towards future prototypes, revisions and deployments, there are a multitude of recommendations and strategies discovered over the course of this project that are outlined below. 

\subsection{Hardware Selection}
A major limitation faced during the design process was the limited of on board memory for the MKRWAN1300. This is entirely dependent on the intended application of the device but features such as displacement measuring had to be cancelled due to the restricted memory. Besides this, the large amount of GPIO pins on the device and availability of Arduino packages do make the device a great choice as a LoRa node.\\\\
It is recommended that a suitable battery system is chosen for future node implementations. Solar powered batteries offer a great way to increase the battery-life of LoRa nodes. Improvements to the software can be made such as smart-sleeping so that the device is not continuously running computations and draining battery. Implementing a system where the default state of the device is sleep, and then waking up after a certain period of time to sample, compute and transmit would vastly increase battery-life. Improving upon these two aspects of battery and software will vastly increase the time between required maintenance for the LoRa nodes.\\\\
The major difficulty in the hardware selection came down to choosing a suitable gateway. Two gateways were provided by the university, one Arduino branded and the other Dragino. After attempting to use each of these gateways in the LoRaWAN system, it was discovered that neither of them were capable of connecting to TNN in Australia. The Arduino gateway was restricted to European frequencies and the Dragino gateway was not capable of LoRaWAN communication even though it was branded as a LoRa capable device. Therefore, the gateway must be carefully selected with respect to the region of deployment and interactivity with TNN or other network integrations. There is an Arduino branded version of the WisGate Edge Lite 2 which offers the same hardware as the gateway used in this project but with added Arduino firmware for easier integration with the Arduino IoT Cloud. This simplifies the setup of the gateway and devices, but it is advised that future deployments avoid using Arduino in general due to the restrictive nature of the network infrastructure. 

\subsection{Network Infrastructure}
The Arduino IoT Cloud offered an easy to use interface for managing IoT things, cloud variables, dashboards and the TNN back-end integration. However, there are numerous bugs that were found during the implementation. The first bug was the use of an incorrect masking string in the things properties header file that is automatically generated. This is a bug because the TNN integration only works in Australia using the AU915 frequency sub band two which is specified when adding the device to the Arduino IoT cloud. The correct mask is `ff000001f000ffff00020000`.\\\\
Another bug found in the Arduno IoT cloud was the inability to download cloud variable history from the dashboard. Since the cloud variables offer up to three months of data retention with the maker subscription, the cloud variable history at any date can be selected and downloaded to a CSV file. However the cloud variables used in this project were unable to be selected. After contacting Arduino they were able to isolate the bug and implement a fix, but not before the data points for the bridge testing were added to a spreadsheet manually.\\\\
Besides the bugs, Arduino does have good support channels and are able to resolve issues with a fast turn around time. This is not the case for TNN. TNN is a community based platform that is run by volunteers and is not particularly beginner friendly. This makes it hard to fix issues, ask for help and stress test LoRa transmission. Therefore it is the recommendation that future research is conducted to find a suitable replacement. Services such as LoRa Server offer open-source components for building LoRaWAN networks and offers features such as adding integrations with different cloud providers. Amazon Web Services (AWS) and Message Queuing Telemetry Transport (MQTT) can be deployed to handle network messaging which offers the possibility of adding custom storage such as a Structured Query Language (SQL) database. This gives more control over the implementation of the LoRaWAN system, and allows for more research into LoRaWAN communication without the same data policy restrictions enforced by TNN and Arduino IoT Cloud. Additionally, deploying this applications onto a local server eliminate the monthly subscription required for Arduino IoT Cloud and offers as much data retention as needed. 

\subsection{Data Analysis and Monitoring}
In addition to acceleration and frequency response, the incorporation of observational analysis to the system would provide a better understanding of the structural responses of the bridge to pedestrian load. For example, the use of a camera with computer vision and machine learning classifiers could be used to count the number of pedestrians on the bridge at different time stamps to give better context to the vibrational responses. The insights from such an implementation can be further used to improve the predictive capabilities of the system.\\\\
The Arduino IoT Cloud dashboard was helpful in monitoring the testing results in real-time, but a more sophisticated network deployment as previously mentioned offers the capability of post-processing and adding a real-time alert system. Post-processing can be combined with observational analysis to remove noise characteristics evident in the frequency response, and the ability to add maximum threshold values in which non-outlier frequency peaks trigger an alert to the Griffith maintenance staff.

\subsection{Antenna and Gateway Upgrades}
As mentioned in the discussion, the limitations of communication range were attributed to the node antenna and gateway. RAK offers a WisGate Edge Pro outdoor gateway that offers a much larger connection range and is capable of handling transmission through more obstacles. An upgraded gateway matched with an upgraded node antenna and an improved network infrastructure can be used to truly unleash the long range capabilities of the LoRaWAN technology. This kind of upgraded infrastructure can support a vast number of LoRa devices which can be used to attain a more accurate vibrational profile of the bridge, and better indicators to the structural health. Upgrading the gateway and network infrastructure allows for channel hopping, which is sending data over multiple channels to handle larger data packets whilst adhering to Australian communication laws. An upgraded gateway can be placed in a strategical location on the roof of a building on the campus, which can be used to facilitate LoRaWAN systems not only on the Griffith footbridge but for numerous applications campus-wide. 

\subsection{Enclosure and PCB Design} 
A key area for improving the enclosure design is the thickness of the lid. The antenna used in this project was too weak to consistently transmit packets through the enclosure and hence the lid was removed for testing. The lid can also be redesigned to accommodate for a new antenna such as offering an opening for transmission. The enclosure can also be made shorter, but would require a redesign of the PCB with smaller dimensions. There is plenty of empty space on the PCB so this would be an easy fix to implement. 

Implementing these recommendations and strategies can result in an advanced campus-wide LoRaWAN IoT deployment that is based on the insights of this project. Moving to more capable hardware for LoRa nodes, antennas and gateways, and more capable network infrastructure can result in a highly beneficial system not only capable of drawing accurate predictive analysis of the structural health of the Griffith footbridge, but also laying the foundation for any other data acquisition needs within the University. By continually understanding the requirements for LoRaWAN deployment and data acquisition, a more reliable, scalable and insightful IoT architecture can be implemented. 

\section{Conclusion}
The extensive testing and analysis of LoRa nodes within the structural health monitoring of the Griffith University footbridge has substantiated the significant potential of LoRaWAN technology in structural health monitoring. This is evident through the accuracy of FFT computations, calibration of the accelerometer, substantial battery life of the prototypes, sensitivity of the data to pedestrian load and the adaptability to noise thresholds.\\\\
The laboratory beam testing demonstrated the successful implementation of FFT computations within the software to identify peak frequencies, which were found to be consistent with the fundamental frequency achieved in the Strand7 FE simulation. The battery life estimation also highlighted the potential for long-term, low-power monitoring which is critical for real-world applications.\\\\
In addition, the bridge testing indicated the sensitivity of the device to changes in pedestrian load, implying that prototype two was effective in monitoring the external influences on the bridge structure, providing insights into the structural behaviour under different load conditions.\\\\
However, there were constraints in terms of communication range and signal strength, primarily attributed to the indoor gateway, node antenna and environmental obstructions. The limited gain of the node antenna, combined with the initial confined indoor placement of the gateway resulted in challenges when attempting to maintain connection at even a small range.\\\\
With these findings it is clear that future wok needs to focus on specific improvements such as moving away from Arduino and TNN, and moving to more reliable, feature-rich alternative private deployments. This shift would allow for greater control over network messaging, local server deployment, eliminating subscriptions and offer more extensive data retention, promoting in house testing and future LoRa research.\\\\
The addition of observational analysis of pedestrian load, improvements in battery technology and software, and upgrades to the enclosure design are recommended for future revisions. Upgrading the node antenna and gateway would also be beneficial to extending the communication range and facilitate channel hopping, allowing for larger data packets and more reliable coverage.\\\\
Overall, the work conducted in this project lays the groundwork for future LoRaWAN-based IoT architecture, specifically in the field of structural health monitoring systems. It underscores the potential and versatility of LoRaWAN technology, but also highlights the need for careful consideration in hardware selection, network infrastructure and sophisticated data analysis methods. With the recommended upgrades, improvements and strategies, the existing system can be evolved into a formidable, high-capacity LoRaWAN network. This advanced infrastructure would not only revolutionize the structural health monitoring of the Griffith University footbridge, but could also potentially transform the entire landscape of large-scale, real-time structural health monitoring, serving as the blueprint for future IoT infrastructures in the domain of structural monitoring. 

