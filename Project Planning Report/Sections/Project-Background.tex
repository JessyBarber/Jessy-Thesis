\section{Project Background}

The Griffith University Gold Coast campus' footbridge is well known in the area as it spans over the busy Gold Coast highway. Used by many students and professors each day, there is a noticeable wobble that leave some speculating about its structural integrity. Many bridges around the world are aging and deteriorating, and maintaining them is becoming an increasingly difficult task. One of the critical indications of the health of a bridge are vibrations which can indicate structural damage or excessive stress \cite{BridgeVibrations}. Traditional methods involve irregular, time-consuming and costly physical inspections that may fail to recognize an issue until it has become severe.\\

Wireless sensor networks (WSNs) using the Long Range Wide Area Network (LoRaWAN) protocol have emerged as a cost effective solution not only for bridge monitoring, but for a wide array of practical applications. The appeal of LoRaWAN is low power consumption, long-range connectivity and its low-cost infrastructure \cite{LoRaAgriculture}. These factors make LoRaWAN an ideal candidate for facilitating an Internet of Things (IoT) system. LoRaWAN technology is versatile in the sense of its modularity especially when using modern carrier boards such as Arduino or Raspberry Pi which support a vast array of sensor modules. For example, LoRaWAN is often used in an agricultural context to monitor soil and crop health \cite{LoRaAgriculture2}. The agricultural applications of LoRaWAN showcase the technology's long distance communication capabilities which is especially useful in rural settings. Additionally, since LoRaWAN typically only sends small packets of data a few times a day, devices on the network offer low power features which can achieve autonomy for up to ten years \cite{LoRaWater}.\\

LoRaWAN is an ideal choice to monitor the health of the Griffith footbridge as sensor nodes can be attached to sections of the bridge to detect and analyze vibrations. The nodes can transmit data wirelessly to a gateway which can be connected to cloud storage or a private database, collecting and aggregating this data for further analysis.\\
Implementing this system will achieve benefits such as real-time monitoring and early detection of problems which will ultimately lead to a reduced maintenance cost. In further research, machine learning algorithms can be used to detect detrimental patterns in the data \cite{MachineL}, allowing engineers to take proactive measures in maintaining the health of the bridge. Ultimately, this project can enhance the safety and reliability of the Griffith footbridge, ensuring that it remains a vital part of the Griffith University's transportation infrastructure for many more years to come. 
