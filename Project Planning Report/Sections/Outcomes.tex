\section{Expected Project Outcomes}
This project involves multiple outcomes that are confined within the scope of the project. The first outcome to meet is a proof of concept that satisfies the high level hardware and software diagrams. This will initially be completed with one sensor node and will achieve the objective of creating a functional IoT system in a closed loop environment. The next outcome to meet is introducing three sensor nodes into the system and writing the software to distinguish between each node's packets. To complete the closed loop testing the nodes will be placed on a test beam set up in the mechanical engineering labs that simulates vibration. Once the closed loop testing has been completed, the permanent implementation of the device will commence. This involves a solar-battery system sufficient of powering the low-powered devices at all times. Theoretical calculations and quantitative testing will be conducted to determine the power drain characteristics of these sensor nodes. Finally, an enclosure will be modeled and printed to house the electronics and power systems and will be used to mount the devices to the bridge. These enclosures also serve the purpose of weather-proofing the sensor nodes. The final outcome is to have a functional IoT system over the length of the Griffith footbridge that is capable of transmitting packets 24/7 to a gateway that will be placed up to 600m away. 
